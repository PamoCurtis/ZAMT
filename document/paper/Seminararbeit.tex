\documentclass[conference]{IEEEtran}
\IEEEoverridecommandlockouts
% The preceding line is only needed to identify funding in the first footnote. If that is unneeded, please comment it out.
%Template version as of 6/27/2024

\usepackage{cite}
\usepackage{amsmath,amssymb,amsfonts}
\usepackage{algorithmic}
\usepackage{graphicx}
\usepackage{textcomp}
\usepackage{xcolor}
\def\BibTeX{{\rm B\kern-.05em{\sc i\kern-.025em b}\kern-.08em
    T\kern-.1667em\lower.7ex\hbox{E}\kern-.125emX}}
\begin{document}

\title{
	SymBioSys 2025: Zelluläre Automaten zur Tumormodellierung
}

\author{\IEEEauthorblockN{Jonas Heuser}
\IEEEauthorblockA{\textit{Fachbereich Informatik} \\
\textit{Hochschule Bonn-Rhein-Sieg}\\
Bonn, NRW, Deutschland \\
jheus2s@smail.inf.h-brs.de}
\and
\IEEEauthorblockN{Patric Mellinghaus}
\IEEEauthorblockA{\textit{Fachbereich Informatik} \\
\textit{Hochschule Bonn-Rhein-Sieg}\\
Bonn, NRW, Deutschland \\
pmell2s@smail.inf.h-brs.de}
\and
\IEEEauthorblockN{Alexander Maul}
\IEEEauthorblockA{\textit{Fachbereich Informatik} \\
\textit{Hochschule Bonn-Rhein-Sieg}\\
Köln, NRW, Deutschland \\
amaul2s@smail.inf.h-brs.de}
}
\maketitle

\begin{abstract}
Diese Seminararbeit behandelt sich mit der wissenschaftlichen Fragestellung, ob und inwiefern man den Prozess des Tumorwachstums und Reaktion des Immunsystems durch zelluläre Automaten modellieren kann.
Zur Beantwortung unserer Fragestellung, haben wir ein Modell innerhalb des frei-zugänglichen NetLogo Programm erstellt und verfeinert und es anschließend mit den Ergebnissen, Modellen und Konklusionen unserer Quellen verglichen.
\end{abstract}

\begin{IEEEkeywords}
zelluläre Automaten, Tumor, Tumorwachstum
\end{IEEEkeywords}

\section{Einführung}
Für unser Projekt-Seminar "SymBioSys" mussten wir uns ein passendes biologisches Forschungsziel zur Simulation und Wissenschaftlichen Ausarbeitung auswählen. Darauf haben wir uns zur Simulation des Wachstums eines abstrakten Tumor innerhalb einer Gewebematrix entschieden, da wir das Forschungsziel interessant fanden und einen nutzen in der Zukunft sahen. Unsere Hauptforschungsfrage war, inwiefern sich unser Modell als Zellulärer Automat vergleichbar mit der Literatur ist. Hinzu kam ein System an sich bewegenden Lymphozyten, die das Immunsystem simulieren sollen.
Daraufhin haben sich die Autoren über den 3 Monaten Ausarbeitungszeit für ca. 3h die Woche zur Weiterentwicklung des Modells zusammengesetzt.


\section{Unser Modell}
\subsection{Simulationsumgebung}
Als Simulationsumgebung wurde das quell-offene Agenten-Simulationspaket NetLogo der Northwestern University Center for Connected Learning and Computer-Based Modeling in Version 6.4.0 benutzt.

\subsection{Modellparameter}

Zum Start werden Globale Konstanten durch den Nutzer definiert:
\begin{itemize}
	\item cell-max-age
	\item healthycells
	\item min-mutated-neighbors
	\item n-lymphos
\end{itemize}

Zudem gibt es folgende globale Variablen
\begin{itemize}
	\item tumor-size \(s\)
	\item amount healthy cells \(n\)
	\item amount mutated cells \(m\)
\end{itemize} 
\vspace{5mm}
Für die Simulation wurden 5 immobile Zelltypen/Agenten konzepiert:
\begin{itemize}
	\item \emph{healthy} 
	\item \emph{mutated}
	\item \emph{dead}
	\item \emph{empty}
	\item \emph{shocked} 
\end{itemize}
mit den Attributen:
\begin{itemize}
	\item \emph{cell-type}
	\item \emph{cell-age}
	\item \emph{cell-density}
	\item \emph{cell-cooldown}
\end{itemize}
\vspace{5mm}
Bei den Lymphozyten handelt es sich um mobile Agenten mit den Attributen:
\begin{itemize}
	\item \emph{age}
	\item \emph{max-age}
	\item \emph{cooldown}
\end{itemize}

\subsection{Modellprozesse}

\subsubsection{setup}
Beim jeden (neu)start einer Simulation werden alle bisherigen Agenten mit den Befehl \emph{clear-all} entfernt und die Ticks zurückgesetzt. Danach werden diese erneut aufgesetzt mit \emph{setup-patches} und \emph{setup-turtles}.
\vspace{5mm}
In \emph{setup-patches} wird für jeden Patch eine zufällige Zahl zwischen 0 und 1 gezogen, welche dann abgeglichen wird mit \emph{healthy-cells}. Ist die Zahl kleiner, wird die \emph{cell-type} auf 'healthy' gesetzt und \emph{cell-age} wird eine zufälliges Alter zwischen 0 und 10 generiert. War die Zahl größer, wird das Patch auf 'empty' gesetzt. Am Schluss wird eine einzige Patch mit \emph{cell-type} = 'mutated' und \emph{cell-age} zwischen 0 und 5 im Zentrum der Matrix initialisiert.
\vspace{5mm}
In \emph{setup-turtles} werden \emph{n-lymphos} generiert, mit einer zufälligen Koordinate, einen \emph{age} = 0 und  \emph{max-age} zwischen 50 und 100.

\subsubsection{Iterationsablauf}

Die Simulation wird gestoppt, wenn alle Patches vom Typ 'gestorben' sind oder keine vom Typen 'mutated' vorhanden sind.
\vspace{5mm}
Zellen vom Typ 'mutated' oder 'shocked' untergehen einen komplexen Mutationsprozess. 'Shocked' Zellen bekommen dabei ihr Attribut \emph{cell-cooldown} jeden Tick dekrementiert, bis bei Wert 0 die Zelle zurück auf den Typ 'mutated' gesetzt wird.

Pro Tick wird zur Kalkulation der Tumorgröße \(tumor-size\) die Gompertz-Funktion: 
\begin{equation*}
	f(t) = a \cdot e^{(-b \cdot e^{(-c \cdot t))}}
\end{equation*} 

mit der Realisierung: 
\begin{math}
	tumor_size(t) =  100 \cdot e ^ {- 3.3 \cdot e ^ {- 0.0054 *\cdot t}}
\end{math}

ausgeführt.

Alle mutierten Zellen haben eine zufällige Chance sich weiter zu vermehren, falls p <= tumor-size.

Dabei wird zufällig eine von jede benachbarte leere Zelle mutiert und zufälliges Alter gegeben mit \(0 \leq cell_age \leq 5\)

Zudem sterben alle benachbarten gesunden Zellen ab, wenn die Anzahl der zu ihr benachbarten mutierten Zellen \(min_mutated_neighbors\) größer gleich ist.

Die Lympho-patches bewegen sich pro Iteration in einen Umkreis von -5° bis 5° nach vorne und führen eine Kollisionsabfrage durch. Lymphos mit Status 'cooldown' bauen pro Iteration jediglich den Cooldown ab.

Bei einer Kollision mit einer mutierten Zelle, wird diese mutierte Zelle als auch alle mutierten Nachbarn in einen Radius 6 um den Lympho getötet und die restlichen überlebenden mutierten Zellen im Radius 2 um jede getötete Zelle bekommen den Status 'shocked' mit \(100 \leq shock_cooldown \leq 120\)

Diese Interaktion stößt die Lymphozyte zurück, lässt Sie schneller altern, und gibt ihr einen cooldown von bis zu 3.

\section{Modelle in der Literatur}

Um unser Modell vergleichen zu können, haben wir Hauptsächlich 4 akademische Quellen ausgewählt. 
\vspace{5mm}
Carlos Valentim et al. untersuchten einen stochastischen zellülaren Automaten für die Tumorvisualisierung. \cite{b1}


Dabei wurden nur Tumorzellen modelliert, die sich in 2 verschiedene Typen aufteilen ließen: Reguläre Tumorzellen (\textbf{RTC}), Stammzellen (\textbf{STC}); 
\vspace{5mm}
RTC besaßten dabei einen Proliferationsfaktor \(p_{max}\), der die maximale Anzahl der Nachkommen bestimmte, wobei für jeder \(p_{max}\) eines Nachkommens gilt: \(p_{max} = p'_{max} - 1\), wobei \(p'_{max}\) der Proliferationsfaktor des Vorgängers.
\vspace{5mm}
Für STC wurden sowohl klonogene als auch echte Stammzellen simuliert. Klonogene können hierbei nur neue RTC bilden, während echte STC neue RTC und STC bilden können.
\vspace{5mm}
Allgemein hatten alle Zellen pro Zeitschritt $\Delta t$ eine Wahrscheinlichkeit \(P_A\) eine Apoptosis durchzugehen, \(P_S\) eine neue Stammzelle zu bilden,\(P_\mu\) in Verhältnis zur Verlagerungskapazität $\mu$ zu migrieren und \(P_P\) sich zu prolifieren.




\section{Vergleich des Modells}

\section{Diskussion und Ausblick}


\subsection{Abbreviations and Acronyms}\label{AA}
Define abbreviations and acronyms the first time they are used in the text, 
even after they have been defined in the abstract. Abbreviations such as 
IEEE, SI, MKS, CGS, ac, dc, and rms do not have to be defined. Do not use 
abbreviations in the title or heads unless they are unavoidable.


\subsection{Equations}
Number equations consecutively. To make your 
equations more compact, you may use the solidus (~/~), the exp function, or 
appropriate exponents. Italicize Roman symbols for quantities and variables, 
but not Greek symbols. Use a long dash rather than a hyphen for a minus 
sign. Punctuate equations with commas or periods when they are part of a 
sentence, as in:
\begin{equation}
a+b=\gamma\label{eq}
\end{equation}

Be sure that the 
symbols in your equation have been defined before or immediately following 
the equation. Use ``\eqref{eq}'', not ``Eq.~\eqref{eq}'' or ``equation \eqref{eq}'', except at 
the beginning of a sentence: ``Equation \eqref{eq} is . . .''

\subsection{\LaTeX-Specific Advice}

Please use ``soft'' (e.g., \verb|\eqref{Eq}|) cross references instead
of ``hard'' references (e.g., \verb|(1)|). That will make it possible
to combine sections, add equations, or change the order of figures or
citations without having to go through the file line by line.

Please don't use the \verb|{eqnarray}| equation environment. Use
\verb|{align}| or \verb|{IEEEeqnarray}| instead. The \verb|{eqnarray}|
environment leaves unsightly spaces around relation symbols.

Please note that the \verb|{subequations}| environment in {\LaTeX}
will increment the main equation counter even when there are no
equation numbers displayed. If you forget that, you might write an
article in which the equation numbers skip from (17) to (20), causing
the copy editors to wonder if you've discovered a new method of
counting.

{\BibTeX} does not work by magic. It doesn't get the bibliographic
data from thin air but from .bib files. If you use {\BibTeX} to produce a
bibliography you must send the .bib files. 

{\LaTeX} can't read your mind. If you assign the same label to a
subsubsection and a table, you might find that Table I has been cross
referenced as Table IV-B3. 

{\LaTeX} does not have precognitive abilities. If you put a
\verb|\label| command before the command that updates the counter it's
supposed to be using, the label will pick up the last counter to be
cross referenced instead. In particular, a \verb|\label| command
should not go before the caption of a figure or a table.

Do not use \verb|\nonumber| inside the \verb|{array}| environment. It
will not stop equation numbers inside \verb|{array}| (there won't be
any anyway) and it might stop a wanted equation number in the
surrounding equation.

\subsection{Some Common Mistakes}\label{SCM}
\begin{itemize}
\item The word ``data'' is plural, not singular.
\item The subscript for the permeability of vacuum $\mu_{0}$, and other common scientific constants, is zero with subscript formatting, not a lowercase letter ``o''.
\item In American English, commas, semicolons, periods, question and exclamation marks are located within quotation marks only when a complete thought or name is cited, such as a title or full quotation. When quotation marks are used, instead of a bold or italic typeface, to highlight a word or phrase, punctuation should appear outside of the quotation marks. A parenthetical phrase or statement at the end of a sentence is punctuated outside of the closing parenthesis (like this). (A parenthetical sentence is punctuated within the parentheses.)
\item A graph within a graph is an ``inset'', not an ``insert''. The word alternatively is preferred to the word ``alternately'' (unless you really mean something that alternates).
\item Do not use the word ``essentially'' to mean ``approximately'' or ``effectively''.
\item In your paper title, if the words ``that uses'' can accurately replace the word ``using'', capitalize the ``u''; if not, keep using lower-cased.
\item Be aware of the different meanings of the homophones ``affect'' and ``effect'', ``complement'' and ``compliment'', ``discreet'' and ``discrete'', ``principal'' and ``principle''.
\item Do not confuse ``imply'' and ``infer''.
\item The prefix ``non'' is not a word; it should be joined to the word it modifies, usually without a hyphen.
\item There is no period after the ``et'' in the Latin abbreviation ``et al.''.
\item The abbreviation ``i.e.'' means ``that is'', and the abbreviation ``e.g.'' means ``for example''.
\end{itemize}
An excellent style manual for science writers is \cite{b7}.

\subsection{Identify the Headings}\label{ITH}
Headings, or heads, are organizational devices that guide the reader through 
your paper. There are two types: component heads and text heads.

Component heads identify the different components of your paper and are not 
topically subordinate to each other. Examples include Acknowledgments and 
References and, for these, the correct style to use is ``Heading 5''. Use 
``figure caption'' for your Figure captions, and ``table head'' for your 
table title. Run-in heads, such as ``Abstract'', will require you to apply a 
style (in this case, italic) in addition to the style provided by the drop 
down menu to differentiate the head from the text.

Text heads organize the topics on a relational, hierarchical basis. For 
example, the paper title is the primary text head because all subsequent 
material relates and elaborates on this one topic. If there are two or more 
sub-topics, the next level head (uppercase Roman numerals) should be used 
and, conversely, if there are not at least two sub-topics, then no subheads 
should be introduced.

\subsection{Figures and Tables}\label{FAT}
\paragraph{Positioning Figures and Tables} Place figures and tables at the top and 
bottom of columns. Avoid placing them in the middle of columns. Large 
figures and tables may span across both columns. Figure captions should be 
below the figures; table heads should appear above the tables. Insert 
figures and tables after they are cited in the text. Use the abbreviation 
``Fig.~\ref{fig}'', even at the beginning of a sentence.

\begin{table}[htbp]
\caption{Table Type Styles}
\begin{center}
\begin{tabular}{|c|c|c|c|}
\hline
\textbf{Table}&\multicolumn{3}{|c|}{\textbf{Table Column Head}} \\
\cline{2-4} 
\textbf{Head} & \textbf{\textit{Table column subhead}}& \textbf{\textit{Subhead}}& \textbf{\textit{Subhead}} \\
\hline
copy& More table copy$^{\mathrm{a}}$& &  \\
\hline
\multicolumn{4}{l}{$^{\mathrm{a}}$Sample of a Table footnote.}
\end{tabular}
\label{tab1}
\end{center}
\end{table}

\begin{figure}[htbp]

\caption{Example of a figure caption.}
\label{fig}
\end{figure}

Figure Labels: Use 8 point Times New Roman for Figure labels. Use words 
rather than symbols or abbreviations when writing Figure axis labels to 
avoid confusing the reader. As an example, write the quantity 
``Magnetization'', or ``Magnetization, M'', not just ``M''. If including 
units in the label, present them within parentheses. Do not label axes only 
with units. In the example, write ``Magnetization (A/m)'' or ``Magnetization 
\{A[m(1)]\}'', not just ``A/m''. Do not label axes with a ratio of 
quantities and units. For example, write ``Temperature (K)'', not 
``Temperature/K''.

\section*{Acknowledgment}

The preferred spelling of the word ``acknowledgment'' in America is without 
an ``e'' after the ``g''. Avoid the stilted expression ``one of us (R. B. 
G.) thanks $\ldots$''. Instead, try ``R. B. G. thanks$\ldots$''. Put sponsor 
acknowledgments in the unnumbered footnote on the first page.

\section*{References}

Please number citations consecutively within brackets \cite{b1}. The 
sentence punctuation follows the bracket \cite{b2}. Refer simply to the reference 
number, as in \cite{b3}---do not use ``Ref. \cite{b3}'' or ``reference \cite{b3}'' except at 
the beginning of a sentence: ``Reference \cite{b3} was the first $\ldots$''

Number footnotes separately in superscripts. Place the actual footnote at 
the bottom of the column in which it was cited. Do not put footnotes in the 
abstract or reference list. Use letters for table footnotes.

Unless there are six authors or more give all authors' names; do not use 
``et al.''. Papers that have not been published, even if they have been 
submitted for publication, should be cited as ``unpublished'' \cite{b4}. Papers 
that have been accepted for publication should be cited as ``in press'' \cite{b5}. 
Capitalize only the first word in a paper title, except for proper nouns and 
element symbols.

For papers published in translation journals, please give the English 
citation first, followed by the original foreign-language citation \cite{b6}.

\begin{thebibliography}{00}
\bibitem{b1} C. Valentim, J. Rabi und S.David, ´´Cellular-automaton model for tumor growth dynamics: Virtualization of different scenarios´´,
Computers in Biology and Medicine, Volume 153, 2023, doi: 10.1016/j.compbiomed.2022.106481
\bibitem{b2} Hatzikirou, H., Breier, G., Deutsch, A. (2020). Cellular Automaton Modeling of Tumor Invasion. In: Sotomayor, M., Pérez-Castrillo, D., Castiglione, F. (eds) Complex Social and Behavioral Systems . Encyclopedia of Complexity and Systems Science Series. Springer, New York, NY.
\bibitem{b3} Hossein Nikravesh Matin, Saeed Setayeshi,
A computational tumor growth model experience based on molecular dynamics point of view using deep cellular automata,
Artificial Intelligence in Medicine,
Volume 148,
2024,
102752,
ISSN 0933-3657,
https://doi.org/10.1016/j.artmed.2023.102752.
\end{thebibliography}

\vspace{12pt}
\end{document}
